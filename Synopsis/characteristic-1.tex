\actuality\
Инструменты, предназначенные для~автоматического поиска дефектов в~программном коде, в~настоящее время получают всё большее распространение. Практика использования инструментов статического и динамического анализа закреплена в ряде методологий разработки безопасного программного обеспечения, таких как Microsoft SDL и TSP-Secure. Активное внедрение инструментов статического анализа объясняется практическим подтверждением их возможностей по обнаружению реальных дефектов в коде программ. 

Различные инструменты сильно различаются по характеристикам анализа: возможностям обнаружения дефектов (полноте), точности и скорости поиска. Задача точного и полного анализа программ на предмет поиска дефектов является алгоритмически неразрешимой, поскольку, согласно теореме Райса, невозможно для произвольной программы доказать её соответствие заданному нетривиальному свойству. В отношении задачи анализа программы это означает, что нельзя в общем случае найти все дефекты заданной программы по произвольно сформулированному критерию. Это приводит к необходимости применения различных ограничений и эвристик, что заставляет искать компромисс между достигаемыми характеристиками. Соответственно, различные методы статического анализа имеют различное соотношение между этими характеристиками.

Для анализа программных систем с высокими требованиями к качеству требуется высокая полнота анализа для обнаружения максимально возможного количества дефектов. Подобные анализаторы должны поддерживать межпроцедурный и межмодульный анализ программ. С увеличением структурной сложности системы возрастает время оценки корректности отчёта о дефекте и, соответственно, стоимость просмотра ложных срабатываний. Это значит, что при анализе сложных многокомпонентных программных систем также требуется высокая точность. Однако алгоритмы анализа с подобными характеристиками имеют экспоненциальную сложность относительно среднего количества операторов в программе, из-за чего они начали получать распространение лишь в последнее время. Множества состояний больших программных систем по определению имеют большую размерность, затрудняющую исследование или моделирование. Анализ больших графов выполнения, имеющихся в больших программных системах, с использованием данных алгоритмов требует длительного времени. Однако статический анализ имеет смысл применять постоянно в процессе разработки. Желательно также иметь возможность проводить анализ <<на лету>>, что также предъявляет повышенные требования к скорости анализа.

При интеграционном тестировании больших программных систем (содержащих большое количество компонент) возникают проблемы, связанные с резким ростом количества возможных путей выполнения программы. Разработанная модификация метода межпроцедурного анализа (МПА) позволяет снизить время, затрачиваемое на анализ путей выполнения программы, и увеличить покрытие графа выполнения в процессе анализа. Всё это приводит к повышению эффективности качества интеграционного тестирования.

Таким образом, повышение производительности межпроцедурного статического анализа имеет важную практическую значимость. Увеличение производительности позволяет осуществлять более полный и точный анализ крупных программ и программных систем, уменьшая время на поиск дефектов, а также позволяет ускорять анализ небольших программ.

Вопросы точного и полного, в т.~ч., межпроцедурного анализа рассматривались различными исследователями. Основополагающей в этой области может считаться работа J.~King (1976~г.), описывающая метод символьного выполнения программы. В~основе метода лежит идея разбиения входных данных на~классы эквивалентности в~зависимости от~встречаемых по пути выполнения условий. Два основных подхода к межпроцедурному анализу сформулированы Amir Pnueli и Micha Sharir. Подход к МПА на основе резюме был применён для смешанного анализа в работах Patrice Godefroid и впоследствии использован для реализации отдельных видов проверок в исследовательских работах Saswat Anand, Koushik Sen и George Necula, Jos\'{e} Miguel Rojas и Corina S. P\u{a}s\u{a}reanu. Попытка использовать резюме для моделирования циклов предпринималась в работах A.~Tsitovich и N.~Sharygina.

Однако данные работы посвящены поиску лишь ограниченного количества дефектов. Кроме того, в работах использован не статический, а смешанный и динамический анализ. В настоящее время требуется подход, который бы обеспечил возможность выполнения проверок произвольного вида, поскольку разработка одноцелевого анализатора является нецелесообразной. Наконец, упомянутые подходы слабо применимы для анализа больших объёмов программного кода из-за большого времени, затрачиваемого на анализ даже небольших программ. Промышленный статический анализатор должен быть многоцелевым, пригодным для решения различных классов задач и иметь возможность реализации широкого класса проверок. В рамках данной работы планировалось построение метода анализа крупных прикладных программных комплексов, разработанных с использованием языков C и C++, способного осуществлять анализ проектов масштаба ОС Android и ОС Tizen (порядка 5--20~млн. строк кода) за приемлемое время и обеспечивающего достаточное покрытие путей выполнения программы. Поэтому для разработки промышленного точного и полного статического анализатора требуется применение иных либо модифицированных подходов.

\aim\ данной работы является теоретическое обоснование и исследование подходов для модификации существующего метода межпроцедурного анализа, который будет использован для построения универсального анализатора кодов программ, разработанных с использованием языков C и C++, для дальнейшего встраивания в среду автоматического тестирования, с целью повышения эффективности анализа крупных прикладных программных комплексов. %построение метода анализа крупных прикладных программных комплексов, разработанных с использованием языков C и C++, способного осуществлять анализ проектов масштаба ОС Android и ОС Tizen \todo{(порядка 5--20~млн. строк кода)} за приемлемое время и обеспечивающего достаточное покрытие путей выполнения программы.

Для достижения указанной цели в диссертационной работе решаются следующие {\tasks}:
\begin{enumerate}
  \item Разработка модификации метода межпроцедурного анализа программ, пригодной для реализации в многоцелевом статическом анализаторе программного кода на языках C и C++ и позволяющей использовать различные виды проверок кода с целью поиска дефектов в нём.
  \item Разработка метода межмодульного анализа программ, реализованных с использованием языков C и C++, для повышения полноты анализа многокомпонентных систем.
  \item Реализация программного обеспечения (многоцелевой анализатор и его утилиты) на основе предложенных методов с целью его промышленного и коммерческого применения для поиска дефектов в исходном коде программных комплексов.
\end{enumerate}

\underline{\textbf{Объектом исследования}} являются методы, алгоритмы и инструменты статического анализа исходных кодов программ.

\underline{\textbf{Предметом исследования}} являются методы межпроцедурного и межмодульного анализа кодов программ, написанных на языке С и С++, их полнота, точность, производительность и масштабируемость.

\underline{\textbf{Соответствие паспорту научной специальности.}} Область исследования соответствует п.~1 <<Модели, методы и алгоритмы проектирования и анализа программ и программных систем, их эквивалентных преобразований, верификации и тестирования>> и п.~2 <<Языки программирования и системы программирования, семантика программ>>.

\underline{\textbf{Методы исследования}} основаны на теоретических положения теории компиляции и анализа программ, теории графов, конечных автоматов, теории множеств.

\novelty

Научная новизна диссертации определяется получением следующих результатов, которые выносятся на защиту:
\begin{enumerate}
  \item Формализованы функциональные требования к разрабатываемому в данной работе методу межпроцедурного анализа.
  \item Разработана модификация метода межпроцедурного анализа программ на основе резюме для метода символьного выполнения для программ, реализованных с использованием языков C и C++. Важными особенностями разработанного метода является поддержка проверок произвольного вида и их одновременного выполнения, а также поддержка модели памяти, используемой в языках C и C++, в том числе, с учётом арифметики указателей, наследования и выравнивания полей структур.
  \item Разработан алгоритм переименования областей памяти для трансляции имён переменных между различными функциями, использующий цепочки доступа. Данный алгоритм используется для установления соответствия между различными объектами, используемыми в функциях.
  \item Формализованы критерии достижимости и отсечения недостижимых ветвей выполнения программы с целью увеличения производительности анализатора при обработке больших графов выполнения (порядка $10^{11}$--$10^{12}$ узлов) в автоматическом режиме, а также устранения ложных срабатываний. Разработан алгоритм построения фрагментов графа выполнения, моделирующих вызов функции, с учётом ограничений на входные данные, известных на момент вызова функции. 
  \item Разработан метод межмодульного анализа программ, реализованных с использованием языков C и C++ для статического анализатора, использующего в качестве входных данных непосредственно исходный код программы. Использование промежуточного представления в виде синтаксического дерева программы позволяет производить анализ без потери информации о программе.
  \item Разработан алгоритм построения отчёта о дефекте при использовании предложенной модификации метода резюме для метода символьного выполнения. Данный метод позволяет строить информативный межпроцедурный отчёт, включающий показ переходов, выполнимых условий и представляющих интерес событий в процессе выполнения программы.
  \item Проведён сравнительный анализ метода МПА, использующего встраивание кода функции, и метода МПА, использующего резюме её эффектов. Результаты автоматического и ручного тестирования, проведённого для данных методов, показали значительное преимущество предложенного метода. Результаты также свидетельствуют об увеличении скорости поиска дефектов: то же самое количество уникальных дефектов можно находить за время, в 2--3 раза меньшее, в сравнении с методом встраивания. Ручная проверка части дефектов показала, что качество анализа при использовании метода резюме не уменьшается в сравнении с методом встраивания и составляет 80--84\%.


\end{enumerate}

\influence\ Разработаны методы анализа программ, применимые для проектов масштаба операционных систем и их наборов пользовательских приложений, реализованные в практически используемом анализаторе программного кода. Предложенные в диссертационной работе методы и алгоритмы позволяют проводить анализ программных систем объёмом порядка 5--20~млн. строк кода (или около $10^{11}$--$10^{12}$ узлов графа выполнения) в автоматизированном режиме. Данные методы и алгоритмы использованы для создания универсального анализатора кодов программ на языках C и C++. Разработан ряд проверяющих модулей с поддержкой предложенного метода межпроцедурного анализа, имеющих высокую и достаточную для практического применения точность анализа.

\underline{\textbf{Теоретическая значимость}} работы состоит в теоретическом обосновании преимущества метода резюме над методом встраивания в отношении полноты и производительности анализа. Формализованы и с помощью построенной математической модели временных затрат на межпроцедурный анализ показаны условия, при которых анализ с использованием метода резюме имеет преимущество перед методом встраивания. В работе показано, что при переходе от МПА с помощью метода встраивания к МПА с помощью метода резюме полнота анализа сохраняется. Показана возможность осуществления межмодульного анализа программ на языках C и C++ без применения компиляции их в промежуточный код.

\reliability\ полученных результатов обеспечивается экспериментальным подтверждением и последующей ручной проверкой отчётов анализатора при анализе исходного кода ОС Android версии 4.2.1. Ряд обнаруженных дефектов может быть найден с использованием других статических анализаторов, например, Coverity SAVE или Clang Static Analyzer (с режимом встраивания). Для тестирования был использован открытый исходный код, а разработанная экспериментальная система помещена в открытый доступ вместе с исходным кодом, что позволяет воспроизвести эксперименты независимо.

% \defpositions
% \begin{enumerate}
%   \item Модификация метода межпроцедурного анализа программ на основе резюме для метода символьного выполнения для программ, реализованных с использованием языков C и C++, с поддержкой проверок произвольного вида и их одновременного выполнения, а также поддержкой модели памяти, используемой в языках C и C++.
%   \item Алгоритм альфа-переименования областей памяти для трансляции имён переменных между различными функциями, использующий цепочки доступа.
%   \item Алгоритм построения фрагментов графа выполнения, моделирующих вызов функции, с учётом ограничений на входные данные, известных на момент вызова функции.
%   \item Метод межмодульного анализа программ, реализованных с использованием языков C и C++, использующий слияние синтаксических деревьев, позволяющий производить анализ без потери информации о программе.
%   \item Алгоритм построения отчёта о дефекте при использовании предложенной модификации метода резюме для метода символьного выполнения.
% \end{enumerate}

\underline{\textbf{Реализация и внедрение результатов работы.}} Результаты диссертационного исследования использованы при разработке программных средств поиска дефектов в составе комплекса статического анализа, используемой подразделениями Samsung Electronics, и используется для анализа исходного кода ПО различного назначения, в частности, мобильных приложений и операционных систем, телевизионного ПО, ПО медицинских систем, и может использоваться для других программных систем, включая настольные приложения и системы управления, с целью поиска потенциальных дефектов, что подтверждается актами о внедрении.

\probation\
Основные результаты работы докладывались~на:
\begin{enumerate}
 \item 10-й Международной Ершовской конференции <<Перспективы систем информатики>> (PSI 2015) (Казань, Россия, 2015)
 \item XII Международной научно-практической конференции <<Инновации на основе информационных и коммуникационных технологий>> (INFO-2015) (Сочи, Россия, 2015)
 \item Открытой конференции по компиляторным технологиям (Россия, Москва, 2015).
\end{enumerate}


\contribution\ Все выносимые на защиту результаты получены лично автором.

\publications\ Основные результаты по теме диссертации изложены в 4 печатных изданиях~\cite{summary-impl-mine,summary-intro-mine,summary-inter-unit-mine,info-2015},
3 из которых изданы в журналах, рекомендованных ВАК~\cite{summary-impl-mine,summary-intro-mine,summary-inter-unit-mine}, 
1~--- в тезисах докладов~\cite{info-2015}. В работах \cite{summary-impl-mine,summary-intro-mine,summary-inter-unit-mine} автору принадлежат теоретические модели, обзорные разделы, описание элементов разработанных методов, а также результаты экспериментального тестирования разработанного в рамках работы ПО.

\textbf{Объем и структура работы.} Диссертация состоит из введения, четырёх глав и заключения. Полный объём диссертации составляет 131 страницу с 18 рисунками и 12 таблицами. Список литературы содержит 71 наименование.


