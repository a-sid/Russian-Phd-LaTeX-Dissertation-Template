%%% Основные сведения %%%
\newcommand{\thesisAuthor}             % Диссертация, ФИО автора
{Сидорин Алексей Васильевич}
\newcommand{\thesisUdk}                % Диссертация, УДК
{004.4'2}
\newcommand{\thesisTitle}              % Диссертация, название
{\MakeUppercase{Метод межпроцедурного и~межмодульного анализа кодов программ, написанных на~языках  С~и~С++, для построения многоцелевого контекстно-чувствительного анализатора}}
\newcommand{\thesisSpecialtyNumber}    % Диссертация, специальность, номер
{05.13.11}
\newcommand{\thesisSpecialtyTitle}     % Диссертация, специальность, название
{Математическое и программное обеспечение вычислительных машин, комплексов и компьютерных сетей}
\newcommand{\thesisDegree}             % Диссертация, научная степень
{кандидата технических наук}
\newcommand{\thesisCity}               % Диссертация, город защиты
{Москва}
\newcommand{\thesisYear}               % Диссертация, год защиты
{2016}
\newcommand{\thesisOrganization}       % Диссертация, организация
{Федеральном государственном бюджетном образовательном учреждении\par высшего профессионального образования\par«Московский государственный технический университет\parимени Н.Э. Баумана»}

\newcommand{\supervisorFio}            % Научный руководитель, ФИО
{Романова Татьяна Николаевна}
\newcommand{\supervisorRegalia}        % Научный руководитель, регалии
{кандидат физико-математических наук, доцент}

\newcommand{\opponentOneFio}           % Оппонент 1, ФИО
{Севостьянов Пётр Алексеевич}
\newcommand{\opponentOneRegalia}       % Оппонент 1, регалии
{доктор технических наук}
\newcommand{\opponentOneJobPlace}      % Оппонент 1, место работы
{кафедры автоматизированных систем обработки информации и управления федерального государственного бюджетного образовательного учреждения высшего профессионального образования «Московский государственный университет дизайна и технологии»}
\newcommand{\opponentOneJobPost}       % Оппонент 1, должность
{профессор}

\newcommand{\opponentTwoFio}           % Оппонент 2, ФИО
{Рудяк Юрий Владимирович}
\newcommand{\opponentTwoRegalia}       % Оппонент 2, регалии
{доктор физико-математических наук}
\newcommand{\opponentTwoJobPlace}      % Оппонент 2, место работы
{кафедры прикладной  математики и моделирования систем  федерального государственного бюджетного образовательного учреждения высшего профессионального образования  «Московский государственный университет печати им. И. Федорова»}
\newcommand{\opponentTwoJobPost}       % Оппонент 2, должность
{профессор}

\newcommand{\leadingOrganizationTitle} % Ведущая организация, дополнительные строки
{\todo{Федеральное государственное бюджетное образовательное учреждение высшего профессионального образования с~длинным длинным длинным длинным названием}}

\newcommand{\defenseDate}              % Защита, дата
{\todo{DD mmmmmmmm YYYY~г.~в~XX часов}}
\newcommand{\defenseCouncilNumber}     % Защита, номер диссертационного совета
{\todo{NN}}
\newcommand{\defenseCouncilTitle}      % Защита, учреждение диссертационного совета
{\todo{Название учреждения}}
\newcommand{\defenseCouncilAddress}    % Защита, адрес учреждение диссертационного совета
{\todo{Адрес}}

\newcommand{\defenseSecretaryFio}      % Секретарь диссертационного совета, ФИО
{\todo{Фамилия Имя Отчество}}
\newcommand{\defenseSecretaryRegalia}  % Секретарь диссертационного совета, регалии
{\todo{д-р~физ.-мат. наук}}            % Для сокращений есть ГОСТы, например: ГОСТ Р 7.0.12-2011 + http://base.garant.ru/179724/#block_30000

\newcommand{\synopsisLibrary}          % Автореферат, название библиотеки
{\todo{Название библиотеки}}
\newcommand{\synopsisDate}             % Автореферат, дата рассылки
{\todo{DD mmmmmmmm YYYY года}}