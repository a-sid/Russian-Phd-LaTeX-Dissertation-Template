\chapter*{Заключение}						% Заголовок
\addcontentsline{toc}{chapter}{Заключение}	% Добавляем его в оглавление

Основные результаты работы заключаются в следующем.
\begin{enumerate}
  \item Формализованы функциональные требования к разрабатываемому в данной работе методу межпроцедурного анализа.
  \item Разработана модификация метода межпроцедурного анализа программ на основе резюме для метода символьного выполнения для программ, реализованных с использованием языков C и C++. Важными особенностями разработанного метода является поддержка проверок произвольного вида и их одновременного выполнения, а также поддержка модели памяти, используемой в языках C и C++, в том числе, с учётом арифметики указателей, наследования и выравнивания полей структур.
  \item Разработан алгоритм переименования областей памяти для трансляции имён переменных между различными функциями, использующий цепочки доступа. Данный алгоритм используется для установления соответствия между различными объектами функций.
  \item Формализованы критерии достижимости и отсечения недостижимых ветвей выполнения программы с целью увеличения производительности анализатора при обработке больших графов выполнения (порядка $10^{11}$--$10^{12}$ узлов) в автоматическом режиме, а также устранения ложных срабатываний. Разработан алгоритм построения фрагментов графа выполнения, моделирующих вызов функции, с учётом ограничений на входные данные, известных на момент вызова функции. 
  \item Разработан метод межмодульного анализа программ, реализованных с использованием языков C и C++ для статического анализатора, использующего в качестве входных данных непосредственно исходный код программы. Использование промежуточного представления в виде синтаксического дерева программы позволяет производить анализ без потери информации о программе.
  \item Разработан алгоритм построения отчёта о дефекте при использовании предложенной модификации метода резюме для метода символьного выполнения. Данный метод позволяет строить информативный межпроцедурный отчёт, включающий показ переходов, выполнимых условий и представляющих интерес событий в процессе выполнения программы.
  \item Проведён сравнительный анализ метода МПА, использующего встраивание кода функции, и метода МПА, использующего резюме её эффектов. Результаты автоматического и ручного тестирования, проведённого для данных методов, показали значительное преимущество предложенного метода. Результаты также свидетельствуют об увеличении скорости поиска дефектов: то же самое количество уникальных дефектов можно находить за время, в 2–3 раза меньшее, в сравнении с методом встраивания. Ручная проверка части дефектов показала, что качество анализа при использовании метода резюме не уменьшается в сравнении с методом встраивания и составляет 80--84\%.
\end{enumerate}

Перспективы развития работы включают в себя следующие направления.
\begin{enumerate}
  \item Моделирование компоновщика в процессе построения дерева сборки с целью увеличения точности анализа.
  \item Повторное использование резюме функции при межмодульном анализе для увеличения производительности анализа с помощью устранения дублирующихся анализов функции.
  \item Дальнейшее улучшение программной реализации предложенных методов анализа.
\end{enumerate}
