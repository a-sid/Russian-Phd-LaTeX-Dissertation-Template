\chapter{Тестирование разработанного программного комплекса}

Разработанная система, реализующая описанные методы, первоначально испытывалась с использованием синтетических тестов, разработанных специально для тестирования корректности реализации. Однако синтетические тесты зачастую не отражают реальное качество анализатора в связи с ограниченностью возможных синтаксических конструкций, а также ввиду ограниченного объёма тестирующего кода. Кроме того, синтетические тесты не позволяют исследовать такие показатели разработанных методов, как масштабируемость и производительность анализатора. Наконец, поскольку разработанный комплекс предполагается к внедрению и промышленному использованию, имеет смысл произвести тестирование на наиболее характерных проектах. В связи с этим возникает необходимость проведения тестирования с использованием кода реальных проектов.

\section{Выбор тестовых проектов}

Для тестирования системы и оценки её количественных и качественных характеристик можно выбрать ряд пакетов различного размера. Желательно также, чтобы среди выбранных проектов присутствовал ряд проектов, не проходивших ранее проверку с использованием известных статических анализаторов. Поскольку после проверки другими статическими анализаторами и исправления найденных ошибок уменьшается количество потенциальных положительных срабатываний, результаты тестирования будут искажены. С другой стороны, вопрос взаимодействия с другими статическими анализаторами также интересен и представляет практический интерес, поскольку для поиска дефектов в разрабатываемом программном коде зачастую используется не один, а несколько анализаторов, причём как статических, так и динамических. Особенный интерес представляет возможность нахождения дефектов после проверки другими анализаторами, поскольку это свидетельствует о возможности дополнения существующих анализаторов или их замены. Это позволит провести сравнение результатов разработанного комплекса на проектах, прошедших статический анализ ранее, и проектах, его не проходивших.

В число интересующих также входят проекты, имеющие отношение к безопасности, такие как \texttt{openssl}, \texttt{openssh}, поскольку обнаруживаемые в них дефекты имеют критическое значение. \todo{Написать ещё}.

Что касается крупных програмных комплексов, то на данную роль была отобрана ОС Android. Данная ОС включает в себя \todo{over 9000} пакетов, связанных между собой, и имеет суммарный объём кода на языках C и C++ около \todo{9000*9000} SLoc. Кроме того, данный программный комплекс включает многие из уже описанных ранее пакетов, что позволяет заменить данным комплексом остальные, которые уже входят в его состав. Особый интерес для тестирования представляет то обстоятельство, что ОС Android может собираться для разных архитектур (x86, x86\_64, ARM и MIPS), причём  исполняемые файлы различных архитектур могут генерироваться во время одной сборки. Большое количество межфайловых связей делают проект интересным для межмодульного анализа и исследования масштабируемости разработанных методов межпроцедурного анализа. Общие характеристики исходного кода ОС Android приведены в таблице \ref{table:android-char}.

\begin{table} [htbp]
  \centering
  \parbox{15cm}{\caption{Характеристики тестовой базы ОС Android}\label{table:android-char}}
%  \begin{center}
  \begin{tabular}{| p{0.4\linewidth} || p{0.5\linewidth} |}
  \hline
  \hline
  Характеристика   & Значение \\
  \hline
  \hline
  Количество строк кода   & \todo{over9000} \\
  \hline
  Количество функций и методов & Регион кода функции   \\
  \hline
  Количество файлов исходного кода      & Регион памяти, располагающийся по адресу, заданному указателю. Не имеет определённого размера и типа, они задаются его подрегионами    \\
  \hline
  Количество транслируемых модулей  & Данные блоковых конструкций языка C и лямбда-выражений C++   \\
  \hline
  Количество архитектур на построение & 2   \\
  \hline
  Количество пакетов & Регион составного литерала \\
  \hline
  \hline
  \end{tabular}
%  \end{center}
\end{table}

\todo{Граф пакетов андроида}

\todo{Граф межпакетных вызовов андроида}

\todo{Диаграма распределения вложенности вызовов}

\section{Методика тестирования}

Для тестирования использовался сервер конфигурации согласно таблице \ref{table:test-server}:

\begin{table} [htbp]
  \centering
  \parbox{15cm}{\caption{Характеристики тестового стенда}\label{table:test-server}}
%  \begin{center}
  \begin{tabular}{| p{0.6\linewidth} || p{0.3\linewidth} |}
  \hline
  \hline
  Характеристика   & Значение \\
  \hline
  \hline
  Модель процессора   & Intel Xeon \todo{over9000} \\
  \hline
  Количество физических процессоров & 2   \\
  \hline
  Количество физических ядер процессора      & 8    \\
  \hline
  Количество виртуальных ядер процессора  & 16 (Hyper-Threading)   \\
  \hline
  Количество виртуальных ядер системы & 32   \\
  \hline
  Объём оперативной памяти & 96 Гб \\
  \hline
  Тип оперативной памяти & DDR3 \\
  \hline
  \hline
  \end{tabular}
%  \end{center}
\end{table}


\section{Тестирование покрытия и производительности}

В качестве критерия производительности при сравнении межпроцедурного анализа методом встраивания и методом резюме можно взять количество узлов графа выполнения, обрабатываемых в единицу времени. 